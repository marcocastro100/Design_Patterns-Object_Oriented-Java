%Packages Padrão:
\documentclass[12pt,a4paper]{article}
\usepackage[latin1]{inputenc}
\usepackage{amsmath}
\usepackage{amsfonts}
\usepackage{amssymb}
\usepackage{makeidx}
\usepackage{graphicx}
%-------------------------------------------------------------------------------------------------------------------------
%Configura as margens com as regras ABNT
\usepackage{geometry}
\geometry{left = 3cm, right = 2cm, top = 3cm,, bottom = 2cm}
%Permite Colocação automática do texto
\usepackage{scrextend}
%-------------------------------------------------------------------------------------------------------------------------
\usepackage{tikz,ifthen,xstring,calc,pgfopts}  %Dependência tikz-uml
\usepackage{tikz-uml} %tikz-uml, arquivo tikz-uml.sty presente no mesmo diretório do trabalho
%-------------------------------------------------------------------------------------------------------------------------
\begin{document}
	

  \begin{tikzpicture}
    \begin{umlseqdiag}

    %Objetos
      \umlactor[fill=white, no ddots]{user}
      \umlboundary [fill=white, x =5, no ddots]{interface}
      \umlentity [fill=white,x = 10, no ddots]{system}
      \umldatabase [x = 15, no ddots, fill=blue!20]{database}
    
    %Chamadas
      \begin{umlcall}[op=Requisi��o de Acesso]{user}{interface}
      
      \begin{umlcall} [op=Req Gera��o Receituario(), return=Req Info Receitu�rio]{interface}{system}
      \end{umlcall}
      \begin{umlcall} [dt = 10,op=Informa Receituario, return=Ret Status de Cadastro]{interface}{system}
            
      \begin{umlcall} [op = Cadastro Receituario(), fill=red!10, return=Retorna Status Cadastro]{system}{database}
      \end{umlcall}
      
           
      \end{umlcall} 
      
      \end{umlcall} % A:B 
      
    \end{umlseqdiag}
  \end{tikzpicture}

\end{document}