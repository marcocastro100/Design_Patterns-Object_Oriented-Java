%Packages Padrão:
\documentclass[8pt,a4paper]{extarticle}
\usepackage[latin1]{inputenc}
\usepackage{amsmath}
\usepackage{amsfonts}
\usepackage{amssymb}
\usepackage{makeidx}
\usepackage{graphicx}
%-------------------------------------------------------------------------------------------------------------------------
%Configura as margens com as regras ABNT
\usepackage{geometry}
\geometry{left = 1cm, right = 1cm, top = 2cm,, bottom = 2cm}
%Permite Colocação automática do texto
\usepackage{scrextend}
%-------------------------------------------------------------------------------------------------------------------------
\usepackage{tikz,ifthen,xstring,calc,pgfopts}  %Dependência tikz-uml
\usepackage{tikz-uml} %tikz-uml, arquivo tikz-uml.sty presente no mesmo diretório do trabalho
%-------------------------------------------------------------------------------------------------------------------------
\begin{document}
  \begin{tikzpicture}
   %FUNCIONARIO
   \umlclass[x = -1, y =3]{Funcionario}{ -- nome : string \\
   	\-- endereco : string \\
   	\-- telefone : string \\
   	\-- e-mail : string \\
   	\-- cpf : string \\
   	\-- totalfuncionarios : int}
   { + editaFuncionario (nome: string, endereco: string, telefone: string,\\ \ e-mail: string, cpf: string )\\        
   \ + cadastrarPaciente(nome: string, endere�o: string, telefone: string,\\ \ email: string, cpf: string)\\    
   \ + editarPaciente(nome: string, endere�o: string, telefone: string,\\ \ email: string, cpf: string)\\
   \ + emitirHistorico (obj{\hspace{.2cm}}agenda: Agenda, obj\_ paciente: Paciente)\\ 
   \ + cadastrarConsulta (paciente: string, medico: string,\\  \ especialidade: string, data: string, horas: string, valor: double,\\ \ obj\_consulta: Consulta: Consulta)\\
   \ + editarConsulta(paciente: string, medico: string, especialidade: string,\\ \ data: string, horas: string, valor: double, obj\_consulta: Consulta)\\
   \ + cancelarConsulta(obj\_consulta: Consulta, obj\_agenda: Agenda)\\
   \ + verificarConsulta(data: string, hora: string, obj\_agenda: Agenda)\\
   \ + listarConsultas(obj\_agenda: Agenda) }
   %SECRETARIO 
   \umlclass[x= 8, y= 0]{Secretario} {}{ + efetuarPagamento(obj\_cliente: cliente,\\ \  formapagamento: string, ncartao: int,\\ \ valor: double)\\ \ + emitirRelatorioMensal()}
   %MEDICO
   \umlclass[x = 8, y=5]{Medico}{-- especialidade: string} { + gerarReceituario(obj\_cliente: Cliente)\\ 
   \ + geralPerfil(obj\_cliente: Cliente)}
   %PACIENTE	
   \umlclass[x= -3, y=-12]{Paciente}{-- nome: string\\ 
   	\-- endere�o: string \\
   	\-- telefone: string \\
   	\-- e-mail: string \\
   	\-- cpf: string \\
   	\-- id: int \\
    \-- totalpacientes: int \\
    \-- historico(): Consulta}{ + Edit(nome: string, endereco: string,\\ \ telefone: string, e-mail: string, cpf: string)\\ 
    \ + editarPaciente(nome: string, endereco: string,\\ \ telefone: string, e-mail: string, cpf: string)}	
    %CONSULTA
    \umlclass[x = 7, y=-7]{Consulta}{-- paciente: string\\ 
       \-- medico: string\\ 
       \-- especialidade: string\\
       \-- data: string\\
       \-- hora: string\\
       \-- valor: double\\
       \-- totalconsultas: int}{ + Consulta(medico: string, paciente: string, especialidade: string,\\ \ data: string, hora: string, valor: double)\\ 
	   \ + editarConsulta(medico: string, paciente: string, especialidade: string\\ \, data: string, hora: string. valor: double)}
    %AGENDA
    \umlclass[x = -1, y=10]{Agenda}{-- vet\_consulta[ ]: Consulta;}{ + adicionarConsulta(obj\_consulta: Consulta,\\ \ obj\_paciente: Paciente\\ 
    \ + cancelarConsulta(obj\_consulta: Consulta)\\ 
    \ + listarConsultas()}
    %SISTEMA
    \umlclass[x= 8, y=10]{Sistema}{}{}
    
    %RELACIONAMENTOS
    \umluniassoc{Secretario}{Funcionario}
    \umluniassoc{Medico}{Funcionario}
    
    \umlassoc{Agenda}{Funcionario}
    \umlassoc{Paciente}{Funcionario}
    \umlassoc{Sistema}{Funcionario}
    
    \umlunicompo{Sistema}{Agenda}
    \umlunicompo{Funcionario}{Consulta}
    \umlunicompo{Paciente}{Consulta}
     \end{tikzpicture}
    
\end{document}